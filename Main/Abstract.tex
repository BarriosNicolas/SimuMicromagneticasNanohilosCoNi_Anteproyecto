\newpage
\thispagestyle{empty}
\section*{Resumen}
En este proyecto de investigación se propone realizar un estudio computacional del estado remanente y de los mecanismos de inversión de la imanación sobre nanohilos individuales de cobalto-níquel ($\ce{Co}$-$\ce{Ni}$) con anisotropía (ejes de fácil imanación transversal al eje del nanohilo). Para este fin, se realizan simulaciones micromagnéticas para estudiar la dependencia de la microestructura remanente en función de la geometría, la anisotropía magnetocristalina y la composición. Además, estudiar los procesos de inversión de imanación (ciclos de histéresis) para condiciones de interés. Uno de los estados más interesantes de este sistema son las cadenas de vórtices magnéticos generadas perpendicular al eje del nanohilo, estructuras magnéticas con potencial interés en aplicaciones de almacenamiento de información, desarrollado de memorias no volátiles, sensores, entre otros.