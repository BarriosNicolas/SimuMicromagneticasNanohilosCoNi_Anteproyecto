\newpage
\section{Introducción}

El micromagnetismo, desarrollado entre $1930$ y $1940$, es una teoría continua que opera en la escala de $1$ nm a varias micras. En dicha escala, la magnetización se puede considerar como una función continua de las coordenadas y su magnitud se mantiene constante, donde la dinámica de la magnetización se rige por las  ecuaciones de Landau-Lifshitz y Gilbert (LLG). Estas ecuaciones nos proporcionan un marco matemático para estudiar como evoluciona la magnetización en el tiempo producto de diversos mecanismos físicos, tanto internos como externos al material. Entre estos se encuentran las interacciones de intercambio, entre dipolos magnéticos y con un campo magnético externo, además de la anisotropía magnetocristalina y de forma \cite{KronmüllerMicromagnetism,Exl2020}.

\vspace{10pt}

Sin embargo, obtener expresiones analíticas para las ecuaciones LLG, salvo en algunas simplificaciones o aproximaciones, es en general una ardua tarea, por lo que las simulaciones micromagnéticas son una herramienta útil para estudiar los sistemas físicos en una escala manométrica. Estas simulaciones consisten, a grandes rasgos, en la resolución numérica de las ecuaciones LLG. Su utilidad radica en que nos permite estudiar en gran detalle los procesos de magnetización de los materiales ferromagnéticos, variando sus propiedades intrínsecas y extrínsecas que experimentalmente sería muy complejo \cite{Tomorrow}. 

\vspace{10pt}

Entre la gran variedad de morfologías nanométricas y subnanométricas que se pueden modelar, los nanohilos cilíndricos ferromagneticos revisten de gran importancia por sus aplicaciones en el campo de almacenamiento de datos, tecnologías de sensores, aplicaciones en espintrónica (diodos tunelados y MRAM), medicina, entre otros. Estos nanohilos son sistemas físicos comúnmente con radios de entre 10 nm a 40 nm que, debido a su pequeño tamaño, presentan propiedades magnéticas exóticas. Estos nanohilos pueden ser obtenidos experimentalmente, por ejemplo, de la electrodeposición en plantillas. Por estas razones, se plantea estudiar el estado remanente y los procesos de inversión de imanación del sistema de nanohilos basados en la aleación $\ce{Co}$-$\ce{Ni}$ con anisotropía transversal al eje del nanohilo. Este sistema se caracteriza por una cadena de vórtices de flujo magnético equidistantes que pueden ser controlados o modulados mediante agentes externos \cite{ExoticMagneticConfiguration}. 