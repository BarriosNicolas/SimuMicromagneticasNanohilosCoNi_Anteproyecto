\newpage
\section{Introducción}

El micromagnetismo, desarrollado entre $1930$ y $1940$, es una teoría continua que opera en la escala de $1$ nm a varias micras. En dicha escala, la magnetización se puede considerar como una función continua de las coordenadas, y su magnitud se mantiene constante, en donde la dinámica de la magnetización se rige por las  ecuaciones de Landau-Lifshitz y Gilbert (LLG). Dichas ecuaciones nos proporcionan un marco matemático para estudiar como evoluciona la magnetización en el tiempo, producto de diferentes procesos físicos tanto internos como externos al material. Entre estos procesos se encuentran las interacciones de intercambio, la interacción de un sistema de partículas con un campo magnético externo, la interacción entre dipolos magnéticos, etc \cite{KronmüllerMicromagnetism,Exl2020}. 
 
%Añadir los procesos de interacción la interacción spin-orbita después de estudiar adetalle la energia de anisotropía magnetocristalina.

\vspace{10pt}

Sin embargo, obtener expresiones analíticas para las ecuaciones LLG, salvo algunas simplificaciones o aproximaciones, es en general una ardua tarea, por lo que las simulaciones micromagnéticas son una herramienta útil para estudiar los sistemas físicos en una escala manométrica. Estas simulaciones consisten, grosso modo, en la resolución numérica de las ecuaciones LLG.  Su utilidad radica en que nos permite estudiar los  procesos de magnetización de los materiales ferromagnéticos en gran detalle variando sus propiedades extrínsecas e intrínsecas, que experimentalmente sería muy complejo \cite{Tomorrow}. 

\vspace{10pt}

Entre la gran variedad de morfologías manométricas y subnanométricas que se pueden modelar, los nanohilos ferromagnéticos revisten de gran importancia por sus aplicaciones en el campo de almacenamiento de datos, tecnologías de sensores, aplicaciones en espintrónica (diodos tunelados y MRAM), medicina entre otros. Los nanohilos ferromagnéticos cilíndricos son sistemas físicos comúnmente con radios de entre 10 nm a 40 nm que, debido a su pequeño tamaño, presentan propiedades magnéticas exóticas. Estos nanohilos pueden ser obtenidos experimentalmente, por ejemplo, de la electrodeposición en plantillas. En este trabajo estudiaremos el estado remanente y los procesos de inversión de imanación del sistema de nanohilos basados en la aleación $\ce{Co}$-$\ce{Ni}$ con anisotropía transversal al eje del nanohilo. Este sistema se caracteriza por una cadena de vórtices de flujo magnético equidistantes la cual puede ser controlada y modulada mediante propiedades intrínsecas y/o extrínsecas \cite{ExoticMagneticConfiguration}.

%En este trabajo estudiaremos el sistema de nanohilos ferromagnéticos basados en la aleación Co-Ni. Este sistema se caracteriza, cuando la magnetización es perpendicular la eje del nanohilo (ocasionado por su alta anisotropía magnetocristalina), por sus estados remanentes magnéticos, los cuales presentan una distribución uniforme y periódica de vórtices (nodos) equidistantes. En donde, el flujo magnético en cada uno de estos nodos rota alrededor de un eje fijo \cite{ExoticMagneticConfiguration}.  