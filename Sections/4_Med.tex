\newpage
\section{Metodología}

Para dar cumplimiento con el $\textbf{OE1}$, se plantean realizar las siguientes actividades.

\vspace{10pt}

\textbf{OE1. Actividad 1.1: Revisión bibliográfica del estudio de nanohilos de Co-Ni.} Se llevará a cabo una revisión bibliográfica centrada en el sistema de nanohilos ferromagnéticos basados en la aleación $\ce{Co}$-$\ce{Ni}$ \cite{CylindricalMagneticNonowires,ExoticMagneticConfiguration,FieldTuneble}. Dicha revisión bibliográfica permitirá determinar la importancia de estudiar este tipo de sistema y conocer que otro tipo de estudio teórico, o de modelamiento, se han desarrollado en los nanohilos de $\ce{Co}$-$\ce{Ni}$. Además de esto, se revisarán aspectos relacionados con el fenómeno micromagnético y la microestructura de materiales ferromagnéticos \cite{Exl2020,KronmüllerMicromagnetism}, las simulaciones micromagnéticas en nanohilos \cite{Kumar_2017,miltat2007numerical} y aspectos generales del magnetismo \cite{jackson2012classical,coey_2010}.

\vspace{10pt}
    
\textbf{OE1. Actividad 1.2: Determinación de las condiciones apropiadas para el desarrollo de simulaciones micromagnéticas en nanohilos de Co-Ni.} De acuerdo con la literatura consultada, se establecerán los parámetros micromagnéticos y rangos apropiados para el sistema de nanohilos con los que se llevarán a cabo las simulaciones. Entre los parámetros relevantes que se deben definir se encuentra la geometría y morfología de los nanohilos; la magnetización de saturación; la constante y tipo de anisotropía y; la constante de intercambio.

\vspace{10pt}

\textbf{OE1. Actividad 1.3: Pruebas en el paquete de simulaciones seleccionado.} Para el desarrollo de las simulaciones micromagnéticas, se evaluarán los posibles métodos o paquetes de simulación con los cuales se podría realizar el trabajo, dando especial importancia en aquellas que sean libres y que permitan simulaciones de la forma más eficiente. Entre los posibles paquetes se encuentra OOMMF, el programa de simulaciones más usado a nivel mundial; y mumax$^3$, software que permite realizar simulaciones usando la GPU.

\vspace{10pt}

\textbf{OE1. Actividad 1.4: Pruebas con los parámetros magnéticos seleccionados.} Teniendo en cuenta los parámetros y rangos establecidos en la actividad 1.2, se realizarán un conjunto de simulaciones en las cuales se variarán los distintos parámetros de interés. En donde se buscará asegurar la validez y confiabilidad de los resultados mediante consideraciones físicas y lo reportado en la literatura.

\vspace{10pt}

\textbf{OE1. Actividad 1.5: Simulaciones del estado remanente de los nanohilos de Co-Ni.} Se realizará una serie de simulaciones del estado remanente de nanohilos de $\ce{Co}$-$\ce{Ni}$. Se realizará un análisis de los estados magnéticos remanentes observados en cada simulación, con el objetivo de identificar patrones y tendencias presentes en el sistema. Además, se buscará establecer correlaciones y proporcionar una explicación física para su causalidad.

\vspace{10pt}

Para dar cumplimiento con el \textbf{OE2}, se plantean realizar las siguientes actividades:

\vspace{10pt}

\textbf{OE2. Actividad 2.1: Construcción del script para realizar simulaciones de ciclos de histéresis de los nanohilos de Co-Ni.} Se desarrollará un script para el paquete de simulación seleccionado en donde se simularán los ciclos de histéresis para el sistema y determinar su comportamiento en diferentes configuraciones.

\newpage

\textbf{OE2. Actividad 2.2: Correlación de los estados magnéticos de los nanohilos de Co-Ni con la dirección de barrios de campo magnético.} Se evaluarán los diferentes estados magnéticos que tome el nanohilo de Co-Ni en función con la orientación del campo magnético aplicado. El objetivo es comprender cómo la orientación del campo magnético afecta la microestructura magnética del nanohilo.