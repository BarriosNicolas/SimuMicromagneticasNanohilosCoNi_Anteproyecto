\newpage
\section{Objetivos}
% Modificar el los verbos empleados "Estudiar"
\subsection{Objetivo general}
Estudiar la microestructura magnética de nanohilos basados en la aleación $\ce{Co}$-$\ce{Ni}$ y su dependencia con variables intrínsecas y extrínsecas.
\subsection{Objetivos específicos}
    \textbf{OE1:} Estudiar la dependencia de la microestructura remanente como función de la geometría del nanohilo, la anisotropía magnetocristalina y la composición.

    \vspace{10pt}

    \textbf{OE2:} Estudiar el proceso de inversión de la imanación para condiciones y parámetros magnéticos de interés.

\subsection{Hipótesis de investigación}
La presencia de vórtices transversales distribuidos a lo largo del eje del nanohilo de CoNi, de manera estable, está vinculada a la consecución de un estado estacionario mediante la minimización de la energía libre de Gibbs en dicho sistema. En este contexto, diversas interacciones de los momentos dipolares magnéticos desempeñan un papel crucial en la configuración y orientación de estos momentos magnéticos, contribuyendo así a la consolidación de una microestructura remanente específica. Dentro de este conjunto de interacciones características de una nanoestructura ferromagnética, variables libres como la temperatura, el tensor de estrés y el campo magnético externo aplicado desempeñan un papel destacado en la energía libre de Gibbs.

\vspace{10pt}

Si consideramos una temperatura constante y la ausencia de deformaciones de la red cuando el sistema se encuentra en la fase cristalográfica HCP (esto es, cuando la concentración de Ni ronda entre el 10$\%$ y el 20$\%$ aproximadamente), cadenas de vórtices transversales se pueden inducir a partir de la aplicación de campos magnéticos externos intensos (campos de saturación) que son luego quitados de forma abrupta para llevar el sistema a remanencia. Creemos que la dirección en la cual se aplique ese campo de saturación es crucial para inducir vórtices transversales en remanencia, y que dicha dirección debe ser perpendicular a la longitud del hilo. Suponemos que pequeñas variaciones en la dirección del campo magnético (con respecto al eje perpendicular del hilo) podrían no ser ideales para la formación de vórtices transversales (desviaciones por encima de 5°) y, por el contrario, se podrían inducir estados vórtices longitudinales. 